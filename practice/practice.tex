% !TeX document-id = {14b11b82-58a9-4271-b498-391391568540}
% !TeX TXS-program:compile = txs:///lualatex/[--shell-escape]

\include{preamble.inc.tex}

\begin{document}
	
\include{title.tex}
\normalsize

\pagenumbering{arabic}
\setcounter{page}{2}

\tableofcontents
\normalsize

\pagebreak

\specsection{ВВЕДЕНИЕ}

Цель производственной практики -- определить, какую часть функционала можно из баннерного демона в отдельный прокси сервер и реализовать сервис для сбора статистики сущностей <<pad>> в рекламной системе.

Задачи практики:
\begin{enumerate}
	\item провести эксперимент по вынесению части функционала из баннерного демона «bannerd» в отдельный прокси сервер, подготовить соответствующие метрики;
	\item по результатам эксперимента определить, какой функционал может быть вынесен, а какой нет;
	\item реализовать инструментарий для аналитики сущностей «pad» инвентаря площадок.
\end{enumerate}

\clearpage

\section{Характеристика предприятия}

ООО «ВК»\cite{vk} было основано в 2000 году. Основной вид деятельности --- создание и использование баз данных и информационных ресурсов. Головной офис расположен в Москве.

В проектах ВК можно общаться, играть, слушать музыку, смотреть и снимать видео, осваивать профессии и навыки, читать новости. 

Для бизнеса ВК развивает продукты и услуги для цифровизации бизнес-процессов — от интернет-продвижения до облачных сервисов.



\clearpage

\section{Описание выполнения задач}

\subsection{Экспериментальный прокси сервера}
В рамках прохождения практики бы реализован прокси сервер, стоящий перед баннерным демоном и производящий рендеринг баннера, без использования результата рендеринга. Исходный код обработчика запросов представлен в листинге \ref{code:handler}.

\begin{lstlisting}[label=code:handler, caption={Обработчик запроса прокси сервер}]
func (h ProxyTemplaterHandler) RequestHandler(ctx *fasthttp.RequestCtx) {
	// request to bannerd
	if err := h.client.Do(&ctx.Request, &ctx.Response); err != nil {
		h.logger.Error("Error in request:", log.Field("err", err))
		sentry.CaptureException(err)
		ctx.Error("Proxy host error", fasthttp.StatusInternalServerError)
		return
	}
	
	// skip result
	_, err := h.render.Render("467", render.Config{
		Db:      &store.Db{},
		Request: &ctx.Request,
	})
	
	if err != nil {
		h.logger.Error("Error in render:", log.Field("err", err))
		sentry.CaptureException(err)
		ctx.Error("Render error", fasthttp.StatusInternalServerError)
		return
	}
	
	ctx.SetStatusCode(fasthttp.StatusOK)
}
\end{lstlisting}

Разность  между средним временем ответа баннерного демона и средним времени ответа прокси сервера представлена на рисунке \ref{fig:metrics}.

\begin{figure}[hbtp]
	\centering
	\includegraphics[width=\textwidth]{inc/metrics}
	\caption{Разность  между средним временем ответа баннерного демона и средним времени ответа прокси сервера.}
	\label{fig:metrics}
\end{figure}

\newpage

Как видно из рисунка, разность составляет порядка 500 наносекунд.  Из этого следует, что прокси сервер увеличивает время ответа на 500 наносекунд с учетом рендеринга.

\subsection{Интерпретация результатов эксперимента}

На данном этапе разработки, совместно с руководителем было принято решение, что пока можно вынести только функционал рендеринга баннера, так как вынесение большего функционала может привести к более долгим ответам сервиса, а также к многочисленным изменения кода в баннерном демоне.

\subsection{Реализация инструментария для аналитики сущностей <<pad>> инвентаря площадок}
Был разработан сервис, который раз в сутки анализирует сущности <<pad>> и распределяет их с помощью определенного алгоритма по <<слоям>> для каждого продукта. Сервис представляет возможность посмотреть слои с $gold$ по $total$, где $gold$ -- считается эталонным инвентарем, а $total$ считается фактическим, оставшиеся -- промежуточные. 

Помимо агрегатов сервис также пишем <<дельты>> в которых описано, что нужно добавить в сущность, чтобы она перешла в следующий <<слой>> . Пример таблицы агрегатов представлен на рисунке \ref{fig:agregates}.

\begin{figure}[hbtp]
	\centering
	\includegraphics[width=\textwidth]{inc/agregates}
	\caption{Таблица агрегатов сущности <<pad>> по слоям.}
	\label{fig:agregates}
\end{figure}
\clearpage

\specsection{Заключение} 
В рамках производственной практики в рекламной системе ООО «ВК» была определена часть функционала для вынесения из баннерного демона в отдельный прокси сервер и реализован сервис для сбора статистики сущностей <<pad>> . Цель практики достигнута. Были выполнены следующие задачи:

\begin{enumerate}
	\item проведен эксперимент по вынесению части функционала из баннерного демона «bannerd» в отдельный прокси сервер, подготовлены соответствующие метрики;
	\item по результатам эксперимента определ  функционал для вынесения;
	\item реализован инструментарий для аналитики сущностей «pad» инвентаря площадок.
\end{enumerate}


\specsection{СПИСОК ИСПОЛЬЗОВАННЫХ ИСТОЧНИКОВ}

\begingroup
\renewcommand{\section}[2]{}
\bibliographystyle{utf8gost705u}
\bibliography{bibliography}   
\endgroup

\end{document}
